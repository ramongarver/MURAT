%%----- Resumen en español e inglés -----%%
\chapter*{}

%% Aplica el estilo de página {empty} solamente a la página actual.
\thispagestyle{empty}

%%----- Título, subtítulo y nombre del autor -----%%
\begin{center}
   {\large\bfseries \myTitleShort: \myTitle}
\end{center}

\begin{center}
    {\myName}
\end{center}
%%------------------------------------------------%%

%%----- Palabras clave en español -----%%
\noindent{\textbf{Palabras clave}: Agente, Sistema multiagente, SMA, Sistema distribuido, Autonomía, Comunicación, Inteligencia Artificial, Semáforo, Tráfico, Control del tráfico urbano}\\
%%-------------------------------------%%

    \vspace{0.7cm}

%%----- Resumen en español -----%%
\noindent{\textbf{Resumen}}\\

Se pretende diseñar un sistema multiagente, denominado \acrshort{murat} (\myTitle), que controle el tráfico en una zona urbana. Este sistema distribuido utilizará varios agentes para gestionar el tráfico a través de semáforos. 

Una parte fundamental del sistema \acrshort{murat} es proporcionar un alto grado de fiabilidad basado, principalmente, en la correcta comunicación y coordinación entre los diferentes agentes que componen el sistema. Esto evitará problemas de colisiones entre vehículos debidas a que existan semáforos que estén en verde de forma errónea en los cruces. Además, permitirá optimizar la temporización de los semáforos con el objetivo de procesar un mayor flujo de tráfico en diferentes escenarios.

Para construir el sistema \acrshort{murat} de forma correcta, se ha de tener en cuenta qué tipos de agentes lo van a constituir y cómo se van a estructurar desde el punto de vista social. La organización de los agentes constituye la base del modelado.

Con el objetivo de comprobar el funcionamiento del sistema \acrshort{murat}, se realizarán varias simulaciones de tráfico. En estas simulaciones, se reproducirán rutas con fuerte congestión de tráfico durante diferentes períodos de tiempo, pretendiendo representar la realidad con el mayor índice de fidelidad posible.
%%------------------------------%%

%% Termina la página actual y hace que la página siguiente sea una página a la derecha (impar), produciendo una página en blanco si es necesario.
\cleardoublepage

\chapter*{}

%% Aplica el estilo de página {empty} solamente a la página actual.
\thispagestyle{empty}

%%----- Título, subtítulo y nombre del autor -----%%
\begin{center}
    {\large\bfseries \myTitleShort: \myTitle}\\
\end{center}

\begin{center}
    \myName\\
\end{center}
%%------------------------------------------------%%

%%----- Palabras clave en inglés -----%%
\noindent{\textbf{Keywords}: Agent, Multiagent system, MAS, Distributed system, Autonomy, Communication, Artificial Intelligence, Traffic light, Traffic, Urban traffic control}\\
%%------------------------------------%%

    \vspace{0.7cm}

%%----- Resumen en inglés -----%%
\noindent{\textbf{Abstract}}\\

The aim is to design a multi-agent system, called \acrshort{murat} (\myTitle), to control traffic in an urban area. This distributed system will use several agents to manage traffic through traffic lights.

A fundamental part of the \acrshort{murat} system is to provide a high degree of reliability based mainly on correct communication and coordination between the different agents that make up the system. This will avoid problems of collisions between vehicles due to traffic lights that are wrongly set to green at intersections. In addition, it will allow optimisation of traffic light timing in order to process a greater flow of traffic in different scenarios. 

In order to build the \acrshort{murat} system correctly, it has to be taken into account which types of agents will constitute it and how they will be structured from a social point of view. The organisation of the agents forms the basis of the modelling.

In order to test the functioning of the \acrshort{murat} system,  several traffic simulations will be carried out. In these simulations, routes with heavy traffic congestion will be reproduced during different periods of time, aiming to represent reality as faithfully as possible.
%%-----------------------------%%
%%------------------------------------------------%%