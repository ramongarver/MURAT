\chapter{Conclusiones}
    \label{chap:eight}
Las evidencias presentadas en las diferentes secciones de la memoria demuestran la consecución de los objetivos propuestos en la sección \ref{section:objetivos}.

A través de la explicación realizada en los apartados \ref{subsection:cruce_simple} y \ref{subsection:cruce_estandar} se ha mostrado la representación de modelos tanto de un cruce simple como de un cruce estándar. Para cada cruce se ha estudiado tanto la topología, la cual se puede ver en las figuras \ref{fig:cruce_simple_topologia} y \ref{fig:cruce_estandar_topologia}, como los diferentes estados, los cuales están expuestos a partir de las figuras \ref{fig:cruce_simple_estados_1y2} y \ref{fig:cruce_estandar_estados_1y2}, respectivamente.

Con respecto a la representación de modelos de ciudad con varios cruces, en la sección \ref{section:cruce_simple2x1} se ha mostrado la representación de un modelo de este tipo. Este modelo no es real y se ha basado en el cruce simple. Para modelar este cruce se ha aumentado el número de cruces simples de forma progresiva. Se ha obtenido como resultado la topología de la figura \ref{fig:cruce_simple2x1_topologia}.

En la sección \ref{section:cruce_real} se ha expuesto la representación de un modelo complejo de ciudad real. Un cruce de Granada en el que se produce la intersección de las calles Avenida de la Constitución, Avenida Doctor Olóriz y Avenida Andaluces. Para modelar este cruce se ha trabajado, en primer lugar, sobre una imagen real del mismo. En dicha imagen, se ha realizado una marcación en base al modelo de representación estudiado previamente, obteniendo el resultado de la figura \ref{fig:cruce_real}. Posteriormente, esta marcación se ha transformado a una representación topológica tal y como se ha hecho con cruces anteriores, véase la representación obtenida en la figura \ref{fig:cruce_real_topologia}.

Mediante la explicación dada en el apartado \ref{subsubsection:adicion_trafico} en relación a la adición de tráfico y en vista a la figura \ref{fig:adicion_trafico_comparativa}, donde se comparan los distintos modos de adición de tráfico al sistema, se evidencia el establecimiento de intervalos pico en base a las diferentes configuraciones y los modos lineal, pico único y doble pico de la simulación .

Por medio de los gráficos mostrados en la sección \ref{section:resultados}, de resultados de la simulación de tráfico en un escenario con varios cruces, se ha demostrado que el número de vehículos que han pasado por el escenario de simulación es mayor si se aplica la política de tiempos variables que si se aplica la política de tiempos fijos. Se ha conseguido aumentar la cantidad de vehículos que puede absorber y procesar un escenario, con respecto a una simulación sin optimización de tiempos, aplicando la política de tiempos variables.


\chapter{Vías futuras}
    \label{chap:nine}
Un sistema que pretende modelar y simular algo tan complejo como la gestión del tráfico, tal y como se ha comentado en varios puntos de esta memoria, no se puede abordar desde un inicio en toda su complejidad. En relación a lo anterior se pueden proponer diferentes trabajos futuros con el objetivo de conseguir un software de mayor complejidad que haga que el sistema sea más fiel a la realidad. Algunas de las posibles acciones futuras son:
\begin{itemize}
    \item\textbf{Obtener datos reales de una ciudad} para realizar simulaciones en relación a los mismos. Se podrían solicitar datos reales de tráfico a un ayuntamiento y, en base a estos, simular las diferentes situaciones que podrían acontecer con el objetivo de verificar si el sistema funciona correctamente.
    \item \textbf{Mejorar la política de asignación de tiempos}. Se podría mejorar o construir un agente inteligente que analizara los datos reales del tráfico diariamente y ajustara la política en función de los mismos. Esto se podría realizar a través de sensores en el cruce y las calles. Sería interesante combinar la capacidad deliberativa, conseguida a través de mecanismos de razonamiento lógico en función del modelo simbólico representado; con la capacidad reactiva, basada en la percepción del entorno en cada instante.
    \item \textbf{Crear una interfaz gráfica que represente la simulación}. Se podría construir una interfaz gráfica que representara la ciudad y mostrara la simulación de tráfico a diferentes velocidades. Sería interesante poder visualizar el escenario de la simulación y el flujo de vehículos durante la misma.
    \item \textbf{Reconocimiento automático de topologías de cruces}. Se podría crear un módulo de reconocimiento de cruces. Estos cruces serían fotografiados por cámaras de 360º y a través de técnicas de visión por computador serían reconocidos y, posteriormente, representados.
    \item \textbf{Implementación de tests y futuros desarrollos basados en \acrshort{tdd}}: Se podría hacer que el código ya implementado tuviera una cobertura total de tests. Además, sería una muy buena idea continuar con el desarrollo siguiendo la metodología de desarrollo guiado por pruebas.
\end{itemize}
En el capítulo \ref{chap:four}, concretamente en la sección referente al planteamiento del problema, sección \ref{section:problem}, se ha expuesto una aproximación genérica al problema a resolver. En ese planteamiento se han establecido algunas simplificaciones para facilitar la construcción del sistema. Sería interesante aumentar la complejidad del sistema reduciendo las simplificaciones planteadas. Algunas de las propuestas son:
\begin{itemize}
    \item \textbf{Tener en cuenta a los peatones}. Los peatones están presentes en casi cualquier entorno. Necesitan interactuar con el tráfico, por ejemplo, a través del semáforo con el objetivo de cruzar la calle. Sería interesante que existiera un sensor de accionamiento manual en semáforos que al ser activado por parte de los peatones indicara al semáforo la necesidad de cambiar a color rojo, permitiendo el paso de  los peatones.
    \item \textbf{Contemplar variables que ajusten el modelo aún más a la realidad}. Algunas de estas variables podrían ser las velocidades de los vehículos, el tiempo que tardan los vehículos en pasar por un cruce, etc.
    \item \textbf{Mejorar y ajustar la adición de vehículos al sistema}. Aunque para los objetivos que se querían lograr haya sido suficiente la adición de vehículos al sistema en base al modelo lineal de picos, ya que lo realmente necesario era generar situaciones de saturación, esta aproximación se podría asemejar aún más a la realidad.
    \item \textbf{Reestructurar el modelo de datos}. Debido al aumento de la complejidad del sistema, sería interesante valorar si el almacenamiento y la representación actual de la información son los más adecuados o no para el mismo. 
\end{itemize}