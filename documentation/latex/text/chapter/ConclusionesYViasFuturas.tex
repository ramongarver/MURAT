\chapter{Conclusiones}
    \label{chap:eight}

\chapter{Vías futuras}
    \label{chap:nine}
Un sistema que pretende modelar y simular algo tan complejo como la gestión del tráfico, tal y como se ha comentado en varios puntos de esta memoria, no se puede abordar desde un inicio en toda su complejidad. En relación a lo anterior se pueden proponer diferentes trabajos futuros con el objetivo de conseguir un software de mayor complejidad que haga que el sistema sea más fiel a la realidad. Algunas de las posibles acciones futuras son:
\begin{itemize}
    \item\textbf{Obtener datos reales de una ciudad} para realizar simulaciones en relación a los mismos. Se podrían solicitar datos reales de tráfico a un ayuntamiento y, en base a estos, simular las diferentes situaciones que podrían acontecer con el objetivo de verificar si el sistema funciona correctamente.
    \item \textbf{Mejorar la política de asignación de tiempos}. Se podría mejorar o construir un agente inteligente que analizara los datos reales del tráfico diariamente y ajustara la política en función de los mismos. Esto se podría realizar a través de sensores en el cruce y las calles. Sería interesante combinar la capacidad deliberativa, conseguida a través de mecanismos de razonamiento lógico en función del modelo simbólico representado; con la capacidad reactiva, basada en la percepción del entorno en cada instante.
    \item \textbf{Crear una interfaz gráfica que represente la simulación}. Se podría construir una interfaz gráfica que representara la ciudad y mostrara la simulación de tráfico a diferentes velocidades. Sería interesante poder visualizar el escenario de la simulación y el flujo de vehículos durante la misma.
    \item \textbf{Reconocimiento automático de topologías de cruces}. Se podría crear un módulo de reconocimiento de cruces. Estos cruces serían fotografiados por cámaras de 360º y a través de técnicas de visión por computador serían reconocidos y, posteriormente, representados.
\end{itemize}
En el capítulo \ref{chap:four}, concretamente en la sección referente al planteamiento del problema, sección \ref{section:problem}, se ha expuesto una aproximación genérica al problema a resolver. En ese planteamiento se han establecido algunas simplificaciones para facilitar la construcción del sistema. Sería interesante aumentar la complejidad del sistema añadiendo las simplificaciones planteadas. Algunas de las propuestas son:
\begin{itemize}
    \item \textbf{Tener en cuenta a los peatones}. Los peatones están presentes en casi cualquier entorno. Necesitan interactuar con el tráfico, por ejemplo, a través del semáforo con el objetivo de cruzar la calle. Sería interesante que existiera un sensor de accionamiento manual en semáforos que al ser activado por parte de los peatones indicara al semáforo la necesidad de cambiar a color rojo, permitiendo el paso de  los peatones.
    \item \textbf{Contemplar variables que ajusten el modelo aún más a la realidad}. Algunas de estas variables podrían ser las velocidades de los vehículos, el tiempo que tardan los vehículos en pasar por un cruce, etc.
    \item \textbf{Mejorar y ajustar la adición de vehículos al sistema}. Aunque para los objetivos que se querían lograr haya sido suficiente la adición de vehículos al sistema en base al modelo lineal de picos, ya que lo realmente necesario era generar situaciones de saturación, esta aproximación no se asemeja demasiado a la realidad.
\end{itemize}