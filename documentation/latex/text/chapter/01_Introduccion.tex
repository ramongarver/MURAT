\part{Motivaciones e introducción}
\chapter{TODO: CapOne}
\section{Introducción}

\section{Motivaciones}

\section{Objetivos}

\section{¿Cómo está organizado este TFG?}
Este trabajo de fin de grado está organizado en x partes, y capítulos e información adicional como: una portada, una autorización para la ubicación del mismo en la biblioteca, una autorización para su defensa, un resumen en español junto a sus palabras clave, un resumen en inglés junto a sus palabras clave, unos agradecimientos, un índice general, un índice de figuras, un índice de tablas y una bibliografía.

Las partes y capítulos están organizados de la siguiente forma:
\begin{itemize}
    \item Parte I: Motivación e introducción. 
    \begin{itemize}
        \item Capítulo 1: Blablabla.
    \end{itemize}

    \item Parte II: Lorem ipsum.
    \begin{itemize}
        \item Capítulo 2: Blablabla.
        \item Capítulo 3: Blablabla.
        \item Capítulo 4: Blablabla.
    \end{itemize}

    \item Parte III: Lorem ipsum.
    \begin{itemize}
        \item Capítulo 5: Blablabla.
        \item Capítulo 6: Blablabla.
        \item Capítulo 7: Blablabla.
    \end{itemize}

    \item Parte IV: Lorem ipsum.
    \begin{itemize}
        \item Capítulo 8: Blablabla.
    \end{itemize}
\end{itemize}

\part{Estado del arte}
\chapter{Sistemas para el control del tráfico en la actualidad}

\part{Construcción del sistema MURAT}
\chapter{Planificación y presupuesto}
\section{Metodología}
\section{Herramientas utilizadas}
\section{Planificación temporal}
\section{Presupuesto}

\chapter{Análisis y diseño}
\section{¿Qué es un agente?}
\subsection{Estructura de un agente}
\subsection{Comunicación entre agentes}

\section{Sociedad de agentes}
\subsection{Agente Semáforo (TrafficLight)}
\subsection{Agente Cruce (Crossroad)}
\subsection{Agente Ciudad (City)}

\section{Modelo de datos de representación de los escenarios}

\chapter{Implementación}

\part{Simulación de tráfico}
\chapter{Escenarios}
\section{Ciudad 1. Cruce simple}
\section{Ciudad 2. Matriz de cruces simples}
\section{Ciudad 3. Cruce estándar}
\section{Ciudad 4. Matriz de cruces estándares}
\section{Ciudad 5. Cruce complejo real}

\chapter{Análisis de los resultados obtenidos}

\part{Conclusiones y vías futuras}
