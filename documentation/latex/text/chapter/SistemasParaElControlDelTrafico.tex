\chapter{Sistemas para el control del tráfico. Una mirada a la historia}
    \label{chap:two}
La evolución de los sistemas de control del tráfico en las calles urbanas se ha producido en relación al desarrollo y uso del automóvil. Después de las guerras mundiales, tras la reactivación de todo el tejido industrial,  el crecimiento del parque automovilístico llevó a la necesidad de construir infraestructuras físicas y crear sistemas del control del tráfico.

El dispositivo físico principal creado para controlar el tráfico es el semáforo. Por ello, el correcto funcionamiento de estos dispositivos determina, en gran medida, la calidad del flujo de tráfico vehicular. Los semáforos aparecieron para controlar el tráfico de ferrocarriles. Inicialmente, se controlaban de forma centralizada en base a un modelo fundamentado en el uso de un temporizador eléctrico automático. La evolución de los sistemas, durante los primeros años, estuvo intrínsecamente ligada a cambios en los sistemas eléctricos.

Posteriormente, aparecieron sistemas que controlaban el tráfico en relación al flujo vehicular a determinadas horas del día. Todavía no se había conseguido un dinamismo real, ni la posibilidad de tener un entorno reactivo, sin embargo, los avances del momento eran considerables para, en un futuro, aprovechar la recopilación de datos con el objetivo mejorar las técnicas de optimización. Hasta este momento, todo estaba construido en base a sistemas de relojes.

Más tarde, apareció un sistema de control analógico por ordenador. Este sistema ajustaba la temporización de la infraestructura en función de la demanda, es decir, de la cantidad de tráfico, obtenida por el sistema a través de detectores físicos.

Entre 1960 y 1970 se produjeron grandes avances. Se construyeron y probaron los primeros sistemas de control de tráfico basados en ordenadores digitales, los cuales poseían cantidades de datos importantes para la época. Estos sistemas, gracias los nuevos conceptos que introdujeron demostraron ser exitosos al reducir los tiempos de espera, los retrasos y los accidentes.

En la década de los setenta la investigación continuó. Se desarrollaron y desplegaron sistemas de control adaptativo como SCOOT y SCATS. Estos sistemas mejoraron considerablemente la gestión del tráfico gracias a una mayor precisión a la hora de adaptar el estado de los semáforos en función del flujo vehicular existente en cada momento. 

Los sistemas de control adaptativos son muy ventajosos con respecto a los sistemas no adaptativos. Esto es debido a que el buen funcionamiento de los sistemas no adaptativos se basa en la actualización frecuente de sus planes de temporización, lo cual es un proceso tedioso. Sin embargo, los sistemas de control adaptativo se ajustan automáticamente a la situación del tráfico en cada momento, permitiendo responder a situaciones anómalas que, en su caso, los sistemas no adaptativos no podrían gestionar.  

Durante la década de los ochenta y principios de la de los noventa hubo una amplia aceptación y adopción de sistemas avanzados de gestión y control de tráfico. El uso de ordenadores se convirtió en el medio aceptado para controlar calles y carreteras, y pronto fue impulsado debido a avances tecnológicos revolucionarios y reducciones del precio de la tecnología. Los avances en microprocesadores eliminaron las limitaciones de rendimiento impuestas por las capacidades de hardware existentes hasta el momento. \cite{evolution}

Hoy en día, las limitaciones al funcionamiento efectivo de los sistemas, a menudo, no son técnicas, sino institucionales o legales. Cada vez existen más leyes en relación a la recopilación, almacenamiento y uso de datos. Esto, en algunos casos, puede dificultar la implementación de sistemas de control de tráfico guiados por datos. \cite{evolution}

Se pueden establecer cinco generaciones en el desarrollo de sistemas de control de tráfico \cite{wang2018review}:
\begin{enumerate}
    \item Primera generación: control no adaptativo del tráfico de forma local en base a datos de flujos de tráfico ya conocidos.
    \item Segunda generación: control adaptativo del tráfico de forma centralizada en función del flujo vehicular en cada instante.
    \item Tercera generación: control adaptativo del tráfico de forma distribuida en función del flujo vehicular de diferentes partes del sistema en cada instante.
    \item Cuarta generación: control adaptativo del tráfico de forma distribuida, incorporando mejoras en el hardware y en el aprovechamiento de las redes móviles, proveyendo un entorno integrado que permita un mejor apoyo para la toma de decisiones.
    \item Quinta generación: control adaptativo del tráfico a través de sistemas basados en un modelo dirigido por datos que usan técnicas de aprendizaje.
\end{enumerate}

En la actualidad, en Israel se está desarrollando un importante programa de carreteras inteligentes. En este programa se utiliza uno de los sistemas de control de tráfico más avanzados en el momento actual, Rekor One Traffic. Este sistema estudia el flujo, los patrones y los volúmenes de tráfico con el objetivo de comprender cómo se desarrolla el mismo para mejorar y optimizar todos sus aspectos posibles subyacentes \cite{rekor}.